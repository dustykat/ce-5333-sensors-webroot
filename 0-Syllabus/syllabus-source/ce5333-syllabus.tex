\documentclass[12pt]{article}
% REVISION NOTES %%%%%%%%%%%%%%%%%%%%%%%%%%%%%%%%%%%%%%%%%%%%
% 2008-0814 Location, Date, Time
% 2008-0814 fixed citations -- added bibliography.
%
%
\usepackage{geometry}                
\geometry{letterpaper}                   
%\geometry{landscape}                
\usepackage[parfill]{parskip}    
\usepackage{daves,fancyhdr,natbib,graphicx,dcolumn,amsmath,lastpage,url}
\usepackage{amsmath,amssymb,epstopdf,longtable}
\usepackage[final]{pdfpages}
\DeclareGraphicsRule{.tif}{png}{.png}{`convert #1 `dirname #1`/`basename #1 .tif`.png}
\pagestyle{fancy}
\lhead{CE 5333 -- DIY Sensor System Integration for Civil Engineers}
\rhead{SPRING 2020}
\lfoot{CE 5333 -- Cleveland}
\cfoot{Page \thepage\ of \pageref{LastPage}}
\rfoot{DRAFT}
\renewcommand\headrulewidth{0pt}
%%%%%%%%%%%%%%%%%%%%%%%%%%%%%%%%%%%%%%%%%%%%%%%%%%%%%%%
\begin{document}
\section*{\center{ { CE 5333 \\ Civil Engineering Sensor Systems Integration} } }
\section*{Time and Location}
Time is listed on attached schedule below. 
This course is an instructor led special-topic course.  There is substantial hands-on component and students need to obtain about \$300 in supplies (listed below) to build the sensor projects.
The projects are mostly water related but the methods are adaptable to other applications.
The tabular schedule is a guideline; we will try to follow it closely, but be prepared to adjust to changes in pace dictated by our collective experience.
\section*{Instructor}
Theodore G. Cleveland, Ph.D., P.E., M. ASCE, F. EWRI\\
Civil Engineering 203F (Texas Tech Office)\\
Cell Phone: 001-832-722-4185  (The 001 is USA Country Code)\\
Email: theodore.cleveland~\texttt{@}ttu.edu\\
\textsl{Copyright $\copyright$ 2020 Theodore G. Cleveland, all rights reserved.} 
\section*{Office Hours}
Open door -- we can meet after each day for questions; also in mornings before/during breakfast\\
\section*{Catalog Description and Prerequisites}
CE 5333: Special Problems in Water Resources (3:3:0). Individual studies in water resources. May be repeated for credit.
\textsl{co-requesite CE 3305} \\
~\\

\section*{Textbook}
Cleveland T. G., DIY Sensors for Civil Engineers  \textsl{in progress}. \\ ~\\ The textbook is located at: \\
\url{http://theodore-odroid.ttu.edu/pending-link}
\section*{Purpose}
A hands-on "maker" course to develop homebrew sensor and data aquisition tools. 
Useful for beginning researchers to collect data at low cost for screening level data.
The sensors and programs are adaptable to professional use.  
Entire systems will be built and demonstrated -- the projects are mostly water related, but concepts are transferable to other Civil sub-disciplines.
Intended for graduate students with minimal electronics skills, and minimal computer administration skills.
Students should know how to solder and have had a circuits class somewhere.

Topics include:
\begin{itemize}
\item System on chip computing (a little history). 
\item         Pratical applications of sensing technology.
\item         Building a sensing and recording system (prototype) using RPi
 \item        Making deployment copies on a RPi-Zero 
\item         Wireless communications using the RPi 3/4 as the controller and the RPi-zero as the slave
 \item        Remote procedures - running the remote from a distance.
 \item        Data processing for interpretation
\item         Introduction to python
\end{itemize}

\section*{Objectives}
Upon completion of this course, students should be able to:
\begin{enumerate}
\item Provision a single-board computer(Raspberry PI; Hardkernel XU4; Arduino, etc) to run a Linux instance and host a wireless network.
\item Construct Analog-to-Digital Converter using MP3008 microprocessor.
\item Attach sensors to system using GPIO pins.
\item Write custom Python scripts to control senors, collect, and process data
\item Use custom built system to control a device based on sensor input.
\item Perform all control activities by remote procedure calls.
\end{enumerate}

\section*{Candidate Projects}
Student teams will build and program a selection of the following example projects
\begin{enumerate}
\item Clean build RPi-3/4 with Ubuntu 19.X operating system (clean image ,not using noobs). Duplicate RPi-Zero
\item A/D converter 8-channel, 10-bit to convert analog signal to digital. Uses MCP3008.
\item Water-level detector, to detect 8 discrete water levels in a vessel (tank, channel, etc.) Uses the A/D converter.
\item Ultrasonic time-of-flight detector.  Use as short range downlooking distance measure tool.
\item Optical laser time-of-flight detector.   Similar application as ultrasonic
\item Hall-detector mass flow meter (counting type).
\item Tipping bucket raingage (counting type). Compare a DIY raingage to adapted flowmeter gage.
\item Gas pressure sensor.  Install in a tensiometer.
\item Normal Stress sensor (similar to strain gage)(possibly improved water level trick)
\item Soil moisture sensor array.  Similar programming as water level sensor.
\item Pump controller circuit to start a small pump based on sensor data. 
\item IR remote temperature detection.
\item Thermal plume tracing using FLIR and image processing. (Advanced project only 2 sensors in inventory)
\end{enumerate}

\newpage
\section*{Course Schedule}
\begin{longtable}{p{0.1in}p{0.8in}p{4.6in}p{1.0in}}
%%%%%%%%
\caption[]{CE 3305 Course Schedule -- Summer 2016
\newline 
\newline
\footnotesize
[ID: Lecture code; each $\approx~$ 1.5 hours in duration;  \\
 DATE \& TIME: Date and time of scheduled lecture; \\
 TOPIC: Lecture content synopsis; \\
 READING: Relevant book pages. \\
} \label{tab:lecture-schedule} \\
\hline
ID & DATE  & TOPIC &  READING \\
\hline
\endfirsthead
\caption[]{CE3305 Schedule and Lecture Abstracts  --- Continued} \\
\hline
ID & DATE \& TIME & TOPIC & READING \\
\hline
\endhead
\hline
\multicolumn{4}{l}{\emph{Continued on next page}}
\endfoot
\hline
\endlastfoot
%%%%%%
%ID & DATE & TOPIC & REMARKS & READING \\
\hline
\hline
%%%%%%%%%%%%%%%%%%%WEEK001%%%%%%%%%%%%%%%%%%%%%%%%%%
\hline
~1 & date & Introduction;  & pp. ~~1 - 25 \\
~2 & date  & Linux images: using Etcher to make an image & pp. ~26 - 43\\
~3 &  date  & Install OS onto RPi; configure network &pp. ~44 - 78 \\
~4 &  date  & Acquire GPIO software (from class website) &pp. ~79 - 86 \\
\hline~5 &  date & Simultaneous Linear Equations & pp. ~87 - 112\\
~6 &  date  & Project 1: AD converter using MCP3008 &  pp. 113 - 119\\
~7 &  date  & Project 1: Test AD converter as a voltmeter & pp. 120 - 125 \\
~8 & date  & Project 2: Water level sensor using resistive sensing & pp. 126 - 133 \\
~9 &  date  & Project 2: Polling the MCP3008 to detect 8-levels of water  & pp. 134 - 144 \\
\hline
10 &  date & Project 3: Hall detector flowmeter (digital counting) & pp. 145 - 153\\
11 &  date &  Project 3: Testing the detector  & pp. XX-XX \\
12 &  date  & Project 4: Gas pressure sensor & pp. XX-XX \\
13 &  date  & Project 4: Testing the sensor & pp. XX-XX\\
\hline
14 &  date  & Project 5: IR remote temperature sensing & pp. XX-XX \\
15 &  date  & Project 6: Pump controller circuit & pp. XX-XX\\
16 &  date  & Final Sensor challenge (Choose remaining projects) & pp. XX-XX \\
17 &  date &  Project 7:  & pp. XX-XX\\
\hline
18 &  date  &  Project 7  & pp. XX-XX\\
19 & 1 date  &  Project 8  & pp. XX-XX \\
\hline
20 &  date  &  Project 8: & pp. XX-XX  \\
21 & date  & Project 9:  & pp. XX-XX \\
22 &  date &  Project 9: & pp. XX-XX \\
23 & date  & Final Demonstrations & pp. XX-XX\\
24 &  date & Alibi Demonstrations& pp. XX-XX\\
\hline
\hline
%\hline
%\end{tabular}
%\clearpage
%%%%%%%%%%%%%%%%%%%%%%%%%%%%%%%%%%%%%%%%%%%%%%%%%%%%
\end{longtable}
\normalsize



%\includepdf[pages=1]{./Summer_School_2015_V07.pdf}
\section*{Assessment Instruments}
\subsection*{Sensor Projects}
Project completion -- all students need functioning projects.  OK to work in teams.
Project reports are comprised of:
\begin{enumerate}
\item Problem statement and sketch of the system
\item Identify and list the  environment and values to be sensed
\item  Identify relevant governing equations and state assumptions
\item Identify sensor components needed.
\item Identify and write control programs to access sensors and acquire data.
\item Demonstrate working project (photograph evidence adequate)
\item Discuss the results
\end{enumerate}
\subsection*{System Documentation}
All projects need documentation of sufficient detail so others can implement.  
The documentation will be in a portfolio (collection of reports for each project) using the above reports.

%Five (5) quizzes are scheduled.\footnote{The quizzes replace the usual mid-term exam.}   These quizzes will be comprised of several problems, possibly verbatim from the homeworks, and conceptual questions derived from lecture and homework materials. 
%\subsection*{Trip Reports}
%Two field trips are scheduled.   
%TTU Students are to write a trip report;  the report is to be in memorandum format no longer than two pages of text.  
%Photographs are encouraged.   
%The reports will be evaluated for content, insight, spelling, and grammar.   
%The reports can be sent by e-mail.\footnote{Jade students are exempt from the trip report requirement.  The field trips will involve the USA students only because the group is too large for everyone to attend.   I will not use trip reports as part of the grade -- except as a completion requirement for the USA students.}

\subsection*{Deployment}
End of semester a deployment scenario that uses one or more projects will be assigned to student teams.  
Students must sucessfully demonstrate their sensor systems to the class.\footnote{If they fail in demonstration, teams will have one week to debug and retry.}
%%%%%%%%%%%%%%%%%%%%%%%%%%%%%%%%%%%%%%%%%%%%%%%
%%%%%%%%%%%%%%%%%%%%%%%%%%%%%%%%%%%%%%%%%%%%%%%
\subsection*{Grading Policy}
Final grades are determined based on performance during the course.
Letter grades will be assigned using University standards.  
The \textbf{approximate} weighting of graded material in determining the final grade is as follows\footnote{Graded materials with fewer than 100 points will have raw scores normalized to 100 points for calculating the final grade.}:

\begin{table}[h!]
   \centering
   \begin{tabular}{l l}
Item & Percent of Grade \\
\hline
\hline
%Trip Reports  & footnote above \\
Projects  Completion   & ~~~~70\% \\
Portfolio Completion      &  ~~~~10\% \\
Deployment Demonstration  & ~~~~20\% \\
\hline
\end{tabular}
\end{table}
%%%%%%%%%%%%%%%%%%%%%%%%%%%%%%%%%%%%%%%%%%%%%%%
\subsection*{ABET Program Outcomes}
A subset of the ABET Program Outcomes are addressed in CE 3305, these outcomes are listed below:\footnote{Item 3[b] below is only partially fulfilled -- in this course students will analyze and interpret data, design of experiments is beyond the scope of the class.}

\begin{tabular}{p{0.5in}p{5.5in}}
\texttt{3[a].}  & Ability to apply knowledge of mathematics, science, and engineering.\\
\texttt{3[b].}  & Ability to design and conduct experiments, as well as to analyze and interpret data.\\
\texttt{3[e].}  & Ability to identify, formulate, and solve engineering problems.\\
\texttt{3[i].}   & Recognition of need for life-long learning.\\
\texttt{3[k].}  & Ability to use the techniques, skills, and modern engineering tools necessary for engineering practice.\\
\texttt{8[d].}  & Proficiency in water resources engineering.\\
\end{tabular}
%%%%%%%%%%%%%%%%%%%%%%%%%%%%%%%%%%%%%%%%%%%%%%%
\subsection*{Academic Misconduct}
Refer to the Texas Tech University Catalog and operating policies (OP 34.12) regarding
academic integrity, cheating, and plagiarism. Academic dishonesty will not be tolerated.
%%%%%%%%%%%%%%%%%%%%%%%%%%%%%%%%%%%%%%%%%%%%%%%
%%%%%%%%%%%%%%%%%%%%%%%%%%%%%%%%%%%%%%%%%%%%%%%
\subsection*{Disability Policy}
\textsl{ ``Any student who, because of a disability, may require special arrangements in order to meet
the course requirements should contact the instructor as soon as possible to make any necessary arrangements.
Students should present appropriate verification from Student Disability Services during the instructors office hours. Please note instructors are not allowed to provide classroom accommodations
to a student until appropriate verification from Student Disability Services has been provided.
For additional information, you may contact the Student Disability Services office at 335 West Hall or
806- 742-2405."}
%%%%%%%%%%%%%%%%%%%%%%%%%%%%%%%%%%%%%%%%%%%%%%%
%%%%%%%%%%%%%%%%%%%%%%%%%%%%%%%%%%%%%%%%%%%%%%%



\bibliographystyle{chicago}	         % (uses file "chicago.bst")
\bibliography{ce3372_spring2010}		% expects file "0-6070_LitRev.bib"
\end{document}


